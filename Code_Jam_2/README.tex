% font typeface overleaf
\documentclass{article}
\title{CodeJAM 2}
\author{Aaryen Mehta}
\usepackage{xcolor}
\usepackage{listings}
\usepackage{graphicx}
\usepackage{amsmath}
\usepackage{hyperref}
\definecolor{codegreen}{rgb}{0,0.6,0}
\definecolor{codegray}{rgb}{0.5,0.5,0.5}
\definecolor{codepurple}{rgb}{0.58,0,0.82}
\definecolor{backcolour}{rgb}{0.95,0.95,0.92}
\lstdefinestyle{mystyle}{
    backgroundcolor=\color{backcolour},   
    commentstyle=\color{codegreen},
    keywordstyle=\color{magenta},
    numberstyle=\tiny\color{codegray},
    stringstyle=\color{codepurple},
    basicstyle=\ttfamily,
    breakatwhitespace=false,         
    breaklines=true,                 
    captionpos=b,                    
    keepspaces=true,                 
    numbers=left,                    
    numbersep=5pt,                  
    showspaces=false,                
    showstringspaces=false,
    showtabs=false,                  
    tabsize=3
}
\lstset{style=mystyle}
\date{}
\begin{document}
\maketitle

\section*{Approach}

\subsection*{Detection of Bots}
$\rightarrow$ Detection of bots is done by detecting \textbf{motion} of the bots.\\
$\rightarrow$ In this approach two consecutive frames are taken and the change in movement is detected.\\
$\rightarrow$ For this, some basic image processing techniques are applied.\\
$\rightarrow$ Such as, \textbf{changing to grayscale, blurring, thresholding, dilation}.\\
$\rightarrow$ Then the contours in the image are detected and they are sorted according to area.\\
$\rightarrow$ And the two largest contours are the ones that are covering the motion of the two bots.\\
$\rightarrow$ Then for the detection of the bot, centroid is marked and tracked.\\
$\rightarrow$ For this \textbf{centroid tracking} is used.\\
$\rightarrow$ Then a bounding rectangle for both contours is found and drawn.\\
$\rightarrow$ And labels, \textbf{bot 1 and bot 2} is put on the centroid coordinate.\\
$\rightarrow$ The centroid, is sadly, selected \textbf{manually}.\\
$\rightarrow$ I tried applying \textbf{template matching, aruco codes and contour based approach} but no effective results were obtained.\\
$\rightarrow$ So I had to go for manually assignment of two centroids to their respective bot numbers.\\

\subsection*{Algorithm}
$\rightarrow$ No specific algorithms used. 

\section*{Results}
$\rightarrow$ Results can be found in this repo in a file named \textbf{output.avi}:\\
$\rightarrow$ Frame rate of the solution is same as the original video, \textbf{24 fps}.\\
$\rightarrow$ The source code can also be found in this repo under \textbf{source.py}.\\
$\rightarrow$ Here is the source code for reference purposes:
\newpage
\begin{lstlisting}[language=python,caption={Source Code}]
from cv2 import cv2
import numpy as np
import math


def dist(x1,y1,x2,y2):
    '''Function to compute Euclidean Distance'''
    return np.linalg.norm(np.array([x2-x1,y2-y1]))


def mind(x1o,y1o,x1n,y1n,x2n,y2n):
    '''This function finds closest point among two new centroid points of two bots''' 
    per = 70/100  #this is the error criteria
    if dist(x1o,y1o,x1n,y1n) > dist(x1o,y1o,x2n,y2n):
        if x1o*(1+per) > x2n and x1o*(1-per) < x2n and y1o*(1+per) > y2n and y1o*(1-per) < y2n:
            return (x2n,y2n)
        else :
            return (x1o,y1o)
    elif dist(x1o,y1o,x1n,y1n) <= dist(x1o,y1o,x2n,y2n):
        if x1o*(1+per) > x1n and x1o*(1-per) < x1n and y1o*(1+per) > y1n and y1o*(1-per) < y1n:
            return (x1n,y1n)
        else :
            return (x1o,y1o)


cap = cv2.VideoCapture('sentry3.mkv')

fourcc = cv2.VideoWriter_fourcc(*'XVID')

out = cv2.VideoWriter('output.avi',fourcc, 24.0, (1440,810))

ret, frame1 = cap.read()
ret, frame2 = cap.read()

count = 0

while cap.isOpened():

    try:
        diff = cv2.absdiff(frame1,frame2)
    except:
        break

    gray = cv2.cvtColor(diff,cv2.COLOR_BGR2GRAY)
    blur = cv2.GaussianBlur(gray,(5,5),0)
    _,thresh = cv2.threshold(blur,20,255,cv2.THRESH_BINARY)
    dilated = cv2.dilate(thresh,None,iterations=3)
    contours,heirarchy = cv2.findContours(dilated,cv2.RETR_TREE,cv2.CHAIN_APPROX_SIMPLE)
    contours = sorted(contours,key=cv2.contourArea,reverse=True)[0:2]

    m2 = cv2.moments(contours[1])
    a,b,c,d = cv2.boundingRect(contours[1])
    x2_n = a+c/2
    y2_n = b+d/2

    m1 = cv2.moments(contours[0])
    a,b,c,d = cv2.boundingRect(contours[0])
    x1_n = a+c/2
    y1_n = b+d/2

    if count == 0 :
        x1_o,x2_o,y1_o,y2_o = x1_n,x2_n,y1_n,y2_n
        count += 1
    else :
        (x1_o,y1_o),(x2_o,y2_o) = mind(x1_o,y1_o,x1_n,y1_n,x2_n,y2_n),mind(x2_o,y2_o,x1_n,y1_n,x2_n,y2_n)
        #print('x1_o=',x1_o,'y1_o=',y1_o,'x2_o=',x2_o,'y2_o=',y2_o,'x1_n=',x1_n,'y1_n=',y1_n,'x2_n=',x2_n,'y2_n=',y2_n)
        if x1_o == x2_o and y1_o == y2_o:
            print()
    #cv2.circle(frame1, (x2_o, y2_o), 5, (0, 0, 255), -1)
    for cnt in (contours):
        (x,y,w,h) = cv2.boundingRect(cnt)

        #if cv2.contourArea(cnt) < 3000 :
        #    continue
        cv2.rectangle(frame1,(x,y),(x+w,y+h),(0,255,0),2)
        cv2.putText(frame1, 'bot 1', (int(x1_o), int(y1_o)+100), cv2.FONT_HERSHEY_SIMPLEX, 0.9, (0,255,0), 2)
        cv2.putText(frame1, 'bot 2', (int(x2_o), int(y2_o)+100), cv2.FONT_HERSHEY_SIMPLEX, 0.9, (0,255,0), 2)

    out.write(frame1)

    cv2.imshow('feed',frame1)

    frame1 = frame2
    ret,frame2 = cap.read()

    if cv2.waitKey(42) == 27:
        break

cap.release()
out.release()
cv2.destroyAllWindows()
\end{lstlisting}
\end{document}